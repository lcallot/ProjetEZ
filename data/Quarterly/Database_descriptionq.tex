\documentclass[11pt,a4paper]{article}
\usepackage[utf8]{inputenc}
\usepackage[francais]{babel}
\usepackage[T1]{fontenc}
\usepackage{amsmath}
\usepackage{amsfonts}
\usepackage{amssymb}
\usepackage{makeidx}
\usepackage{graphicx}
\usepackage{tabularx}
\usepackage{multirow}
\usepackage{booktabs}
\usepackage{array}


\usepackage[left=1cm,right=1cm,top=3cm,bottom=2cm]{geometry}
\title{Note on the Quarterly Database}
\author{clément carrier}


\begin{document}
\maketitle

\section{Database}

In this file, I describe the database used, and explain in more details the transformation made before using them into our model. If not mentioned, variables are introduce for the 11 original members countries of the eurozone + Greece + Sweden + UK + Denmark.\\

All the variable in the folder /IMF, /ECB and /OECD contain .csv file corresponding to original (raw) database. We transform the variables thanks to the .R file called "transformation". Then we save subsets from the complete database according to the model that we want to estimate. 




\subsection{ECB}

These are the variables from ECB data warehouse originally in monthly frequency (I took the mean on each quarter) : 
\begin{itemize}
\item HICP : ICP.M.COUNTRY.N.000000.4.INX
\item Processed food exc. Alcohol and tobacco : ICP.M.COUNTRY.N.FDPXAT.4.INX
\item Unprocessed food	: ICP.M.COUNTRY.N.FOODUN.4.INX
\item Indus good exc. Energy : ICP.M.COUNTRY.N.IGXE00.4.INX
\item Energy	:	ICP.M.COUNTRY.N.NRGY00.4.INX
\item Services	:	ICP.M.COUNTRY.N.SERV00.4.INX
\item Unemployment : STS.M.COUNTRY.S.UNEH.RTT000.4.000
\end{itemize}

\vspace{0.5cm}

These variables are loaded directly from a file called ecbq.RData in the subfolder data/Quarterly/ECB. This file was built from the .R file used for the monthly frequency database. \\

I also use in quarterly frequency : 

\begin{itemize}
\item Compensation per employee : $MNA.Q.N.COUNTRY.W2.S1.S1._Z.COM_PS._Z._T._Z.IX.V.N$
\item Exchange rate EUR/USD : EXR.Q.USD.EUR.SP00.A
\item Nominal Effective Exchange Rate : EXR.Q.E1.EUR.ERC0.A
\item GDP : $MNA.Q.N.COUNTRY.W2.S1.S1.B.B1GQ._Z._Z._Z.XDC.LR.N$
\item Productivity : $MNA.Q.N.COUNTRY.W0.S1.S1._Z.LPR_PS._Z._T._Z.IX.LR.N$
\item DowJones Euro Stoxx 50 index : FM.Q.U2.EUR.DS.EI.DJES50I.HSTA

\end{itemize}

\vspace{0.5cm}

These variables are in the folder Data/ECB and are raw variables directly loaded from the ECB database. These variables are transformed in the .R file called "traitementECB" located in the subfolder Data/ECB. (The transformation are describe in the following table) \\

Two databases are saved in the subfolder Data/ECB, one in level, the other one differentiated one time: 

\begin{itemize}
\item ecb.Rdata 
\item difecb.RData 
\end{itemize}

\vspace{0.5cm}

In this file ("traitementECB.R"), I also transform the monthly frequency database into quarterly  frequency in order to add variable to the quarterly database when not available in quarterly frequency. This file is saved in the subfolder /Data/Quarterly/ECB under the name "ecbq.RData" 






\subsection{IMF}

These are the variables from IMF (IFS) :
\begin{itemize}
\item Commodity Prices, Crude Oil, Petroleum (World)
\item London Interbank Offer Rate, 3-Month (US)
\item M1 (Eurozone)
\item M3 (Eurozone)
\item Interest Rates, Monetary Policy-Related Interest Rate, FPOLM (Eurozone + US)
\item Central Bank, Total Gross Assets (ECB + FED)
\item Nominal Effective Exchange Rate, Consumer Price Index (Eurozone)
\item Industrial Production
\end{itemize}

\vspace{0.5cm}

These variable are in the folder Data/IMF and are raw variable directly loaded from the IFS database. These variables are transformed in the .R file called "traitementIMF" located in the subfolder Data/IMF. (The transformation are describe in the following table) \\

Two databases are saved in the subfolder Data/IMF, one in level, the other one differentiated one time: 

\begin{itemize}
\item imf.Rdata 
\item difimf.RData 
\end{itemize}

\vspace{0.5cm}

In this file ("traitementIMF.R"), I also transform the monthly frequency database into quarterly  frequency in order to add variable to the quarterly database when not available in quarterly frequency. This file is saved in the subfolder /Data/Quarterly/IMF under the name "imfq.RData" 






\subsection{OECD}

These variable is in the folder /OECD and are raw variables directly loaded from the OECD database.  These variables are transformed in the .R file called "traitementOECD" located in the subfolder Data/OECD. (The transformation are describe in the following table) \\


These are the variables from OECD :
\begin{itemize}
\item Short-term Interest rates (Monthly Monetary and Financial Statistics)
\item Long-term Interest rates (Monthly Monetary and Financial Statistics)
\end{itemize}

\vspace{0.5cm}

In this file ("traitementOECD.R"), I also transform the monthly frequency database into quarterly  frequency in order to add variable to the quarterly database when not available in quarterly frequency. This file is saved in the subfolder /Data/Quarterly/OECD under the name "oecdq.RData" 




\section{Recapitulation table}



\begin{center}
\begin{table}[h]
 \centering
  \scalebox{0.76}{
    \begin{tabular}{crrrrrrr}
    \toprule
    \multicolumn{5}{c}{Database}
      \\
      \toprule
       &
      Variable  &
      Source &
      Country &
      Adjustment &
      Original &
      Frequency &
      Transformation 
      
      \\
    \midrule
    \multirow{7}[14]{*}{Price} &
      HICP  &
      ECB &
      by country &
      NSA/NWD &
      index &
      monthly&
      log
      \\
     &
      Processed food exc. Alc \& tob &
      ECB &
      by country &
      NSA/NWD &
      index &
      monthly &
      log
      \\
     &
      Unprocessed food &
      ECB &
      by country &
      NSA/NWD &
      index &
      monthly &
      log
      \\
     &
      Indus good exc. Energy &
      ECB &
      by country &
      NSA/NWD &
      index &
      monthly &
      log
      \\
     &
      Energy &
      ECB &
      by country &
      NSA/NWD &
      index &
      monthly &
      log
      \\
     &
      Services &
      ECB &
      by country &
      NSA/NWD &
      index &
      monthly &
      log
      \\
     &
      Oil Price &
      IFS (IMF) &
      World &
      NSA&
      index &
      monthly &
      log
      \\
      \midrule
    \multirow{2}[4]{*}{Economic Activity} &
      Indus. Production &
      IFS (IMF) &
      by country &
      NSA&
      index &
      monthly &
      log
      \\
      &
      Unemployment rate &
      ECB &
      by country &
      SA/NWD&
      \% & 
      monthly &
      /100
      \\
      \midrule
    \multirow{3}[6]{*}{Interest rates} &
      Short-term Interest rates &
      OECD &
      by country &
      NSA/NWD &
      annual \% & 
      monthly &
      /100
      \\
     &
      Long-term Interest rates &
      OECD &
      by country &
      NSA/NWD &
      annual \% & 
      monthly &
      /100
      \\
     &
      LIBOR 3 month &
      IFS (IMF) &
      US &
      NSA/NWD &
      annual \% &
      monthly &
      /100
      \\
      \midrule
    \multirow{4}[8]{*}{Monetary} &
      M1 &
      IFS (IMF) &
      EZ &
      NSA/NWD &
      EUR &
      monthly &
      log
      \\
     &
      M3 &
      IFS (IMF) &
      EZ &
      NSA/NWD &
      EUR &
      monthly &
      log
      \\
     &
      Monetary Policy-Related Rate &
      IFS (IMF) &
      US/EZ &
      NSA/NWD &
      annual \% &
      monthly &
      /100
      \\
     &
      Central Bank Asset &
      IFS (IMF) &
      US/EZ &
      NSA/NWD &
      USD/EUR &
      monthly &
      log
      \\
      \midrule
    \multirow{2}[4]{*}{Other} &
      NEER &
      IFS (IMF) &
      EZ &
      NSA/NWD &
      index &
      monthly &
      log
      \\
     &
      EUR/USD exchange rate &
      ECB &
      EZ &
      NSA/NWD &
      1 EUR = x USD &
      monthly &
      -
      
      \\
    \bottomrule
    \end{tabular}
    }

\end{table}
\end{center}


\end{document}