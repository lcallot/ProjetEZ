\documentclass[11pt,oneside, a4paper]{amsart}\usepackage[]{graphicx}\usepackage[]{color}
%% maxwidth is the original width if it is less than linewidth
%% otherwise use linewidth (to make sure the graphics do not exceed the margin)
\makeatletter
\def\maxwidth{ %
  \ifdim\Gin@nat@width>\linewidth
    \linewidth
  \else
    \Gin@nat@width
  \fi
}
\makeatother

\definecolor{fgcolor}{rgb}{0.345, 0.345, 0.345}
\newcommand{\hlnum}[1]{\textcolor[rgb]{0.686,0.059,0.569}{#1}}%
\newcommand{\hlstr}[1]{\textcolor[rgb]{0.192,0.494,0.8}{#1}}%
\newcommand{\hlcom}[1]{\textcolor[rgb]{0.678,0.584,0.686}{\textit{#1}}}%
\newcommand{\hlopt}[1]{\textcolor[rgb]{0,0,0}{#1}}%
\newcommand{\hlstd}[1]{\textcolor[rgb]{0.345,0.345,0.345}{#1}}%
\newcommand{\hlkwa}[1]{\textcolor[rgb]{0.161,0.373,0.58}{\textbf{#1}}}%
\newcommand{\hlkwb}[1]{\textcolor[rgb]{0.69,0.353,0.396}{#1}}%
\newcommand{\hlkwc}[1]{\textcolor[rgb]{0.333,0.667,0.333}{#1}}%
\newcommand{\hlkwd}[1]{\textcolor[rgb]{0.737,0.353,0.396}{\textbf{#1}}}%

\usepackage{framed}
\makeatletter
\newenvironment{kframe}{%
 \def\at@end@of@kframe{}%
 \ifinner\ifhmode%
  \def\at@end@of@kframe{\end{minipage}}%
  \begin{minipage}{\columnwidth}%
 \fi\fi%
 \def\FrameCommand##1{\hskip\@totalleftmargin \hskip-\fboxsep
 \colorbox{shadecolor}{##1}\hskip-\fboxsep
     % There is no \\@totalrightmargin, so:
     \hskip-\linewidth \hskip-\@totalleftmargin \hskip\columnwidth}%
 \MakeFramed {\advance\hsize-\width
   \@totalleftmargin\z@ \linewidth\hsize
   \@setminipage}}%
 {\par\unskip\endMakeFramed%
 \at@end@of@kframe}
\makeatother

\definecolor{shadecolor}{rgb}{.97, .97, .97}
\definecolor{messagecolor}{rgb}{0, 0, 0}
\definecolor{warningcolor}{rgb}{1, 0, 1}
\definecolor{errorcolor}{rgb}{1, 0, 0}
\newenvironment{knitrout}{}{} % an empty environment to be redefined in TeX

\usepackage{alltt}
\usepackage{natbib}

\usepackage{amsbsy,amsmath}
\usepackage{amssymb,amsfonts}
\usepackage{bbm}%give 1 with dbl vertical bar 
\usepackage{booktabs,url,enumerate}
\usepackage{color,xcolor,colortbl}
\usepackage{float}
\usepackage{tikz}
\usepackage{rotating,graphicx,lscape}
\usepackage{commath}
\usetikzlibrary{arrows,positioning} 
\usepackage[hypcap]{caption}
\newcommand{\sgn}{\mathrm{sign}}
\usepackage{setspace}

% bold rows
\usepackage{array}
\newcolumntype{$}{>{\global\let\currentrowstyle\relax}}
\newcolumntype{^}{>{\currentrowstyle}}
\newcommand{\rowstyle}[1]{\gdef\currentrowstyle{#1}%
  #1\ignorespaces
}

% Invisible table columns!
\newcolumntype{H}{>{\setbox0=\hbox\bgroup}c<{\egroup}@{}}% Properly placed sideways table with asmart class. 

\setlength\rotFPtop{0pt plus 1fil} 


\usepackage[top=1.5cm, bottom=1.5cm, left=3.0cm, right=3.0cm]{geometry}

\DeclareMathOperator{\Med}{\mathbb{M}ed}
\DeclareMathOperator{\Mean}{\mathbb{M}ean}
\DeclareMathOperator{\Cov}{\mathbb{C}ov}
\DeclareMathOperator{\Var}{\mathbb{V}ar}
\DeclareMathOperator{\E}{\mathbb{E}}
\DeclareMathOperator{\nid}{NID}
\DeclareMathOperator{\N}{\mathcal{N}}
\DeclareMathOperator{\corr}{corr}
\DeclareMathOperator{\diag}{diag}
\onehalfspace


\definecolor{LightRed}{rgb}{1,.88,.88}
\definecolor{LightBlue}{rgb}{.88,.88,1}
\definecolor{LightGreen}{rgb}{.88,1,.88}

\newtheorem{theorem}{Theorem}
\IfFileExists{upquote.sty}{\usepackage{upquote}}{}
\begin{document}

  
\title{Bootstrap Results}   
\author{Clement Carrier}
\date{\today}
\maketitle


\section*{Bootstrap}










We simulate the data by choosing, the sparsity of the true parameters (number of non zero coefficient), the number of covariates, the number of observations and the nature of the noise (here we choose iid N(0,1)). 







% latex table generated in R 3.2.1 by xtable 1.7-4 package
% Fri Aug  7 11:00:59 2015
\begin{table}[ht]
\centering
\caption{Simulation Result} 
\label{Test_table}
{\footnotesize
\scalebox{1}{
\begin{tabular}{|r|r|rrrrr|rrrrr|}
  \toprule 
   beta=0.9   &  & & &  iid1 & &  & &  AR(1) & & & \\
 \midrule 
Model & X.p.n. & lenght & cov1 & fn & biais1 & biais2 & lenght2 & cov1.1 & fn2 & biais1.1 & biais2.1 \\ 
    1 & (50,100) & 0.427 & 0.345 & 33.386 & 0.273 & 0.330 & 0.259 & 0.138 & 3.399 & 0.216 & 0.162 \\ 
     2 & (100,100) & 0.394 & 0.386 & 49.198 & 0.294 & 0.383 & 0.273 & 0.079 & 10.163 & 0.263 & 0.195 \\ 
     3 & (200,100) & 0.360 & 0.432 & 63.837 & 0.312 & 0.429 & 0.271 & 0.047 & 20.786 & 0.299 & 0.227 \\ 
   \bottomrule 
\end{tabular}
}
}
\end{table}





In the following one, we increase the number of nonzero parameter (5 and 10). 

\begin{knitrout}
\definecolor{shadecolor}{rgb}{0.969, 0.969, 0.969}\color{fgcolor}\begin{kframe}


{\ttfamily\noindent\color{warningcolor}{\#\# Warning in readChar(con, 5L, useBytes = TRUE): impossible d'ouvrir le fichier compressé 'data/2/mc2.Rdata', cause probable : 'No such file or directory'}}

{\ttfamily\noindent\bfseries\color{errorcolor}{\#\# Error in readChar(con, 5L, useBytes = TRUE): impossible d'ouvrir la connexion}}

{\ttfamily\noindent\color{warningcolor}{\#\# Warning in readChar(con, 5L, useBytes = TRUE): impossible d'ouvrir le fichier compressé 'data/2/mc3.Rdata', cause probable : 'No such file or directory'}}

{\ttfamily\noindent\bfseries\color{errorcolor}{\#\# Error in readChar(con, 5L, useBytes = TRUE): impossible d'ouvrir la connexion}}\end{kframe}
\end{knitrout}


% latex table generated in R 3.2.1 by xtable 1.7-4 package
% Fri Aug  7 11:00:59 2015
\begin{table}[ht]
\centering
\caption{Simulation Result 2} 
\label{Test_table}
{\footnotesize
\scalebox{1}{
\begin{tabular}{|r|r|rrrrr|rrrrr|}
  \toprule 
      &  & & &  iid5 & beta =1 &  & &  iid10 & beta =1 & n=200 & \\
 \midrule 
Model & X.p.n. & lenght & cov1 & fn & biais1 & biais2 & lenght2 & cov1.1 & fn2 & biais1.1 & biais2.1 \\ 
    1 & (50,100) & 0.452 & 0.564 & 263.582 & 0.272 & 0.476 & 0.223 & 0.300 & 905.668 & 0.230 & 0.709 \\ 
     2 & (100,100) & 0.394 & 0.386 & 49.198 & 0.294 & 0.383 & 0.027 & 0.038 & 990.599 & 0.328 & 0.666 \\ 
     3 & (200,100) & 0.360 & 0.432 & 63.837 & 0.312 & 0.429 & 0.003 & 0.002 & 999.000 & 0.849 & 0.150 \\ 
   \bottomrule 
\end{tabular}
}
}
\end{table}


I chose a bigger beta in order to overcome the bias issue. 




% latex table generated in R 3.2.1 by xtable 1.7-4 package
% Fri Aug  7 11:00:59 2015
\begin{table}[ht]
\centering
\caption{Simulation Result 2} 
\label{Test_table}
{\footnotesize
\scalebox{1}{
\begin{tabular}{|r|r|rrrrr|rrrrr|}
  \toprule 
      &  & & &  iid5 & beta =5 &  & &  iid10 & beta =5 &  & \\
 \midrule 
Model & X.p.n. & lenght & cov1 & fn & biais1 & biais2 & lenght2 & cov1.1 & fn2 & biais1.1 & biais2.1 \\ 
    1 & (50,100) & 0.423 & 0.287 & 0.000 & 0.279 & 0.313 & 0.474 & 0.421 & 0.000 & 0.263 & 0.320 \\ 
     2 & (100,100) & 0.460 & 0.212 & 0.000 & 0.331 & 0.396 & 0.708 & 0.356 & 10.837 & 0.356 & 0.533 \\ 
     3 & (200,100) & 0.499 & 0.162 & 0.285 & 0.376 & 0.481 & 1.887 & 0.586 & 241.878 & 0.450 & 1.536 \\ 
   \bottomrule 
\end{tabular}
}
}
\end{table}









% latex table generated in R 3.2.1 by xtable 1.7-4 package
% Fri Aug  7 11:00:59 2015
\begin{table}[ht]
\centering
\caption{Simulation Result} 
\label{Test_table}
{\footnotesize
\scalebox{1}{
\begin{tabular}{|r|r|rrrrr|rrrrr|}
  \toprule 
      &  & & beta=5 &  iid5 & n=400  &  & &  AR(3) &  & & \\
 \midrule 
Model & X.p.n. & lenght & cov1 & fn & biais1 & biais2 & lenght2 & cov1.1 & fn2 & biais1.1 & biais2.1 \\ 
    1 & (50,100) & 0.179 & 0.260 & 0.000 & 0.125 & 0.132 & 0.005 & 0.000 & 200.000 & 1.615 & 0.002 \\ 
     2 & (100,100) & 0.182 & 0.166 & 0.000 & 0.145 & 0.151 & 0.004 & 0.000 & 200.000 & 1.614 & 0.001 \\ 
     3 & (200,100) & 0.183 & 0.111 & 0.000 & 0.161 & 0.170 & 0.004 & 0.000 & 200.000 & 1.614 & 0.001 \\ 
   \bottomrule 
\end{tabular}
}
}
\end{table}




Then we compute the method used by lahiri (On the residual empirical process based on the ALASSO in high dimensions and its functional oracle property). In this paper, Lahiri uses the ALASSO estimator and shows that the empirical distribution of estimated residual behaves approximately as a gaussian noise. He then deduces a confidence band of prediction of the variable of interest (y) based on the empirical distribution of the residual.





% latex table generated in R 3.2.1 by xtable 1.7-4 package
% Fri Aug  7 11:01:00 2015
\begin{table}[ht]
\centering
\caption{Simulation Result} 
\label{Test_table}
{\footnotesize
\begin{tabular}{|r|r|rr|rr|rr|}
  \toprule 
    &  & iid1 &  & iid5 & & AR & \\
 \midrule 
Model & X.pn. & coverage & lenght & coverage.1 & lenght.1 & coverage.2 & lenght.2 \\ 
    1 & (10,100) & 0.881 & 3.454 & 0.938 & 3.675 & 0.923 & 3.712 \\ 
     2 & (50,100) & 0.978 & 3.868 & 0.988 & 4.213 & 0.773 & 3.029 \\ 
     3 & (120,100) & 1.000 & 4.320 & 0.904 & 3.799 & 0.879 & 3.447 \\ 
     4 & (200,100) & 0.907 & 3.557 & 0.907 & 3.555 & 0.977 & 4.111 \\ 
   \bottomrule 
\end{tabular}
}
\end{table}


\end{document}
